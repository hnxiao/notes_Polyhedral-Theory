
\documentclass{article}

\setlength{\oddsidemargin}{0.05 in}
\setlength{\evensidemargin}{-0.05 in}
\setlength{\topmargin}{-0.6 in}
\setlength{\textwidth}{6.5 in}
\setlength{\textheight}{9.5 in}
\setlength{\headsep}{0.25 in}
\setlength{\parskip}{0.1 in}

%
% ADD PACKAGES here:
\usepackage [usenames] {color}
\definecolor {infocolor} {rgb} {0.6,0.6,0.6}
\definecolor {steel blue}{rgb}{0.274510,0.509804,0.705882}
\everymath{\color{steel blue}}
\everydisplay{\color{steel blue}}
%

\usepackage{amsmath,amsfonts,amssymb,enumerate,graphicx}

%
% The following commands set up the lecnum (lecture number)
% counter and make various numbering schemes work relative
% to the lecture number.
%
\newcounter{lecnum}
\renewcommand{\thepage}{\thelecnum-\arabic{page}}
\renewcommand{\thesection}{\thelecnum.\arabic{section}}
\renewcommand{\theequation}{\thelecnum.\arabic{equation}}
\renewcommand{\thefigure}{\thelecnum.\arabic{figure}}
\renewcommand{\thetable}{\thelecnum.\arabic{table}}

\DeclareMathOperator{\rk}{rk}

\newcommand{\tran}{^{\mbox{\tiny $\top$}}}
\newcommand{\tleq}{^{\mbox{\tiny $\leqslant$}}}
\newcommand{\teq}{^{\mbox{\tiny $=$}}}

%
% The following macro is used to generate the header.
%
\newcommand{\lecture}[2]{
   \pagestyle{myheadings}
   \thispagestyle{plain}
   \newpage
   \setcounter{lecnum}{#1}
   \setcounter{page}{1}
   \noindent
   \begin{center}
       \vbox{\vspace{2mm}
         \hbox {\leftline{\Large LECTURE #1: \hfill}}
         \vspace{3mm}
         \hbox {\leftline{\Large #2 \hfill}}
         \vspace{4mm}
         \hrule
         \vspace{3mm}
         \hbox to 6.5in { {{\large Polyhedral Theory}  \hfill April 2013} }
         \vspace{3mm}
        }
   \end{center}
   \markboth{LECTURE #1: #2}{LECTURE #1: #2}
   \pagenumbering{arabic}
   \vspace*{4mm}
}
%
% Convention for citations is authors' initials followed by the year.
% For example, to cite a paper by Leighton and Maggs you would type
% \cite{LM89}, and to cite a paper by Strassen you would type \cite{S69}.
% (To avoid bibliography problems, for now we redefine the \cite command.)
% Also commands that create a suitable format for the reference list.
\renewcommand{\cite}[1]{[#1]}
\def\beginrefs{\begin{list}%
        {[\arabic{equation}]}{\usecounter{equation}
         \setlength{\leftmargin}{2.0truecm}\setlength{\labelsep}{0.4truecm}%
         \setlength{\labelwidth}{1.6truecm}}}
\def\endrefs{\end{list}}
\def\bibentry#1{\item[\hbox{[#1]}]}

%Use this command for a figure; it puts a figure in wherever you want it.
%usage: \fig{NUMBER}{SPACE-IN-INCHES}{CAPTION}
\newcommand{\fig}[3]{
        	\vspace{#2}
			\begin{center}
			Figure \thelecnum.#1:~#3
			\end{center}
	}
% Use these for theorems, lemmas, proofs, etc.
\newtheorem{theorem}{Theorem}[lecnum]
\newtheorem{lemma}[theorem]{Lemma}
\newtheorem{proposition}[theorem]{Proposition}
\newtheorem{claim}[theorem]{Claim}
\newtheorem{corollary}[theorem]{Corollary}
\newtheorem{definition}[theorem]{Definition}
\newenvironment{proof}{{\it Proof.}}{ \hfill $\square$}

% **** IF YOU WANT TO DEFINE ADDITIONAL MACROS FOR YOURSELF, PUT THEM HERE:

\def\R{{\mathbb R}}
\def\Q{{\mathbb Q}}

\begin{document}
%FILL IN THE RIGHT INFO.
%\lecture{**LECTURE-NUMBER**}{**DATE**}{**LECTURER**}{**SCRIBE**}
\lecture{1}{Fundamental concepts on polyhedra}
%\footnotetext{These notes are partially based on those of Nigel Mansell.}

% **** YOUR NOTES GO HERE:

% Some general latex examples and examples making use of the
% macros follow.
%**** IN GENERAL, BE BRIEF. LONG SCRIBE NOTES, NO MATTER HOW WELL WRITTEN,
%**** ARE NEVER READ BY ANYBODY.

The reason why we put the fundamental theorem of linear inequalities, decomposition theorem of polyhedra, and Farkas' lemma in the same place is that: the following those results are the corollaries of `` The fundamental theorem of linear inequalities''. So the fundamental theorem of linear inequality is the key throughout this notes. This points can not be overemphasized.

The fundamental theorem of linear inequalities is an typical example of theorem of alternatives.

\section{The Fundamental theorem of linear inequalities}

pivoting rule proof;
Fourier-Motzkin elimination proof


\section{Cones, polyhedra, and polytopes}

We restrict ourselves to Euclidean space throughout our discussion of polyhedral theory?

Before we can actually start with discrete geometry we need to review some material from linear algebra and fix some terminology.

Let $x_1,\dots,x_k\in \R^n$, and $\lambda_1,\dots,\lambda_k\in \R$. The $\sum_{i=1}^{k}\lambda_i x_i$ is called a \emph{\textbf{linear combination}} of vectors $x_1,\dots,x_k$. It is further a (i) \emph{\textbf{conical combination}}, if $\lambda_i\geq 0$; (ii)  \emph{\textbf{affine combination}}, if $\sum_{i=1}^{k}\lambda_i =1$; and a
(iii) \emph{\textbf{convex combination}}, if it is both conical and affine.

The  \emph{\textbf{linear}}(\emph{\textbf{conical}}, \emph{\textbf{affine}},  \emph{\textbf{convex}})  \emph{\textbf{hull}} of a set $X\subseteq \R^n$ is the set of vectors that  are linear(or, conical, affine, convex, respectively) combinations of vectors in $X$. It is denoted by $\mbox{lin}(X)$(or, $\mbox{cone}(X)$, $\mbox{aff}(X)$, $\mbox{conv}(X)$, respectively). $X$ is a  \emph{\textbf{linear space}}(\emph{\textbf{cone}},  \emph{\textbf{affine space}},  \emph{\textbf{convex set}}) if $X$ equals its linear hull(or, conical hull, affine hull, convex hull, respectively).

For any non-zero vector $a\in\R^n$ and $\beta\in \R$, the set $\{x~|~a^T x\leqslant 0\}$ is a  \emph{\textbf{linear half-space}}, and $\{x|a^T x\leqslant \beta\}$ is an \emph{\textbf{affine half-space}}. Their boundaries $\{x|a^T x=0\}$ and $\{x|a^T x=\beta\}$ are a  \emph{\textbf{linear}} and  \emph{\textbf{affine hyperplane}} respectively.

We can now define the basic objects that we will study in polyhedral theory.

A  \emph{\textbf{polyhedral cone}} is a cone $C\subseteq\R^n$ of the form
\begin{equation*}
C:=\{x~|~Ax\leqslant 0\}
\end{equation*}
for some matrix $A$, i.e., $C$ is the intersection of \emph{finitely} many linear half-spaces. 

Recall that a cone $C:=\mbox{cone}(X)$ is the conical combinations of vectors in $X$. We say vectors in $X$ \emph{\textbf{generate}} cone $C$.
A cone arising in this way is called  \emph{\textbf{finitely generated}}.

It now follows from the Fundamental theorem that for cones the concepts of `polyhedral' and `finitely generated' are equivalent.

It is intuitively obviously that the concepts of polyhedron and of polytope are related. This is made more precise in the Decomposition theorem for polyhedra and in its direct corollary, the Finite basis theorem for polytopes.
\begin{definition} For two subsets $P$, $Q\subseteq \R^n$, the set 
\begin{equation*}
P+Q:=\{p+q~|~p\in P, q\in Q\}
\end{equation*}
is the \emph{\textbf{M{\scriptsize INKOWSKI}  sum}} of $P$ and $Q$.
\end{definition}

\begin{proposition}
Minkowski sums of polyhedra are polyhedra.
\end{proposition}
\begin{proof}
cf Paffenholz, "Polyhedral Geometry and lInear optimization", proposition 2.11
\end{proof}

It follows from the Fundamental theorem that for cones the concepts of `polyhedral' and `finitely generated' are equivalent.

\begin{theorem}[F{\scriptsize ARKAS}-M{\scriptsize INKOWSKI}-W{\scriptsize EYL} Theorem]
A convex cone is polyhedral if and only if it is finitely generated.
\end{theorem}

\emph{\textbf{Finite basis theorem for polyhedral cones}}.

\begin{theorem}[M{\scriptsize INKOWSKI}-S{\scriptsize TEINITZ}-W{\scriptsize EYL} Theorem]
A subset $P\subseteq\R^n$ is a polytope if and only i $P$ is a bounded polyhedron.
\end{theorem}

\emph{\textbf{Finite basis theorem for polytopes}}.

A map $f:\R^n \rightarrow \R^m$ is an \emph{\textbf{affine map}} if there is a matrix $M\in\R^{m\times n}$ and a vector $t\in\R^m$ such that $f(x)=Mx+t$. 

\begin{proposition}
Affine images of polyhedra are polyhedra.
\end{proposition}
\begin{proof}
cf Paffenholz, "Polyhedral Geometry and lInear optimization", proposition 2.11
\end{proof}

Here we always assume affine map $f$ is surjective. In particular, if $f$ is not injective, we refer to it as an \emph{\textbf{projection}}. We have considered a special projection map: Projection along a coordinates axis (with m=n-1), in the Fourier-Motzkin elimination proof of The Fundamental theorem of linear inequalities. In the following reverse procedure of Homogenization, we will still see that special affine transformation.

Following is a very simple but useful fact about projection:
\begin{proposition}
Let $f:P\rightarrow Q$ be a projection of polyhedra. Then for any face $F$ of $Q$, the preimage $f^{-1}(F)=\{x\in P~|~f(x)\in Q \}$ is a face of $P$. Furthermore, if $F$, $G$ are faces of $Q$, then $F\subseteq Q$ holds if and only if $f^{-1}(F)\subseteq f^{-1}(G)$.
\end{proposition}
\begin{proof}
cf lecture in polytopes
\end{proof}

\emph{\textbf{Remark}}: Up to an affine map we can always assume that a polyhedron is full-dimensional. This is a very useful property. Geometrically, if a polyhedron is not full-dimensional, it is embedded in an affine hyperplane or an affine space. So one can always assume the polyhedron is full-dimensional Since by projection, one can obtain an full-dimensional polyhedron in a lower dimensional space with all information preserved. By induction on the sum of dimension of the space and dimension of the polyhedron, desired results can be derived. A explicit example: Proof of Theorem 2 in ``On cutting planes'' by Schrijver.

\section{Homogenization}
In our notation system,  polyhedral cones are defined to be special polyhedra with right hand side $b=0$. In this section we shall show that they are in a sense as general as polyhedra. Indeed we shall construct for a given polyhedron $P\subseteq \R^n$ a polyhedral cone $C$ which still contains all information about $P$. Needless to say that polyhedral cones usually are much easier to handle than general polyhedra. Therefore this construction principle, the so called ``homogenization'', together with the corresponding inverse operation, the so called ``dehomogenization" are useful tools in polyhedral theory.

We associated a polyhedron $P\subseteq \R^n$ with a polyhedral cone $C\subseteq\R^{n+1}$ by embedding $P\subseteq\R^n$ into a hyperplane $x_{n+1}=1$ of $\R^{n+1}$ and letting $P$ be the intersection of polyhedral cone $C$ with hyperplane $x_{n+1}=1$ in $\R^{n+1}$.

Mathematically, let $P\subseteq \R^n$ be a polyhedron. Then the set
\begin{equation}
\mbox{homog}(P)=\{\lambda\begin{pmatrix}x\\1\end{pmatrix}: x\in P, \lambda\in \R_+\}+\{\begin{pmatrix}x\\0\end{pmatrix}:x\in\mbox{char.cone}(P)\}
\end{equation}
is called the \emph{\textbf{homogenization}} of $P$.

Alternatively, $\mbox{homog}(P)$ can be represented by
\begin{equation}
\mbox{homog}(P)=\{\begin{pmatrix}x\\x_{n+1}\end{pmatrix}\in\R^{n+1}:Ax\leq x_{n+1}b,x_{n+1}\geqslant 0\}
\end{equation}

The following result provides an explicit description of $\mbox{homog}(P)$ in terms of a system of linear inequalities.

\begin{proposition}
Let $P=P(A,b)$ be a polyhedron in $\R^n$. Then the homogenization of $P$ is $P(B,0)$, where
\begin{equation}
B:=\begin{pmatrix}A & -b \\ 0 & -1\end{pmatrix}\in \R^{(m+1)\times(n+1)}.
\end{equation}
\end{proposition}
\begin{proof}
\end{proof}

H-representation for homogenization

$\mbox{homog}(P)$ can be represented by
\begin{equation}
\mbox{homog}(P)=\{\begin{pmatrix}x\\x_{n+1}\end{pmatrix}\in\R^{n+1}:Ax\leq x_{n+1}b,x_{n+1}\geqslant 0\}
\end{equation}
Thus, we can see that $\mbox{homog(P)}$ is the H-cone given by the system of inequalities 
\begin{equation}
\begin{pmatrix}x\\x_{n+1}\end{pmatrix}\begin{pmatrix}x\\x_{n+1}\end{pmatrix}\leqslant \begin{pmatrix}0\\0\end{pmatrix}
\end{equation}
and that (do we need the following part?)
\begin{equation}
P=\mbox{homog}(P)\cap \{x\in\R^{n+1}~|~x_{n+1}=1\}
\end{equation}
Conversely, if $C$ is any H-cone in $K^{n+1}$(in fact, any H-polyhedron), it is clear that $P=C\cap\ \{x~|~x_{n+1}=1\}$ is an H-polyhedron in $K^n$.


Inner representation.

Let $P\subseteq\R^n$ is a V-polyhedron, $P=\mbox{conv}(X)+\mbox{cone}(Y$, where $X=\{x_1,\dots,x_p\}$ and $Y=\{y_1,\dots, y_q\}$.

Define $\hat{X}=\{\hat{x}_1,\dots,\hat{x}_p\}\subseteq\R^{n+1}$ and $\hat{Y}=\{\hat{y}_1,\dots,\hat{y}_p\}\subseteq\R^{n+1}$, by
$\hat{x}_i=\begin{pmatrix}x_i\\1\end{pmatrix}$, $\hat{y}_j=\begin{pmatrix}y_j\\0\end{pmatrix}$.
We check immediately that 
$\mbox{homog}(P)=\mbox{cone}(\hat{X}\cup\hat{Y})$ is a V-cone in $\R^{n+1}$ and such that 
$P=\mbox{homog}(P)\cap H_{n+1}$.

Conversely, if $C=\mbox{cone}(W)$ is a V-cone in $K^{n+1}$, with $w_{n+1}\geqslant 0$ for every $w_{n+1}\in W$($w_{i,n+1}\geqslant 0$ for every $w_{i,n+1}\in W$), then $P=C\cap H_{n+1}$ is a V-polyhedron in $\R^n$.

Homogenization and de-homogenization actually reveal the correspondence between polyhedron and cone.


\section{Decomposition theorem for polyhedra}
\begin{theorem}[Decomposition theorem for polyhedra]
A subset $P\subseteq \R^n$ is a polyhedron if and only if $P=Q+C$ for some polytope $Q$ and some polyhedral cone $C$.
\end{theorem}

\section*{References}
\beginrefs
\bibentry{CW87}{\sc D.~Coppersmith} and {\sc S.~Winograd},
``Matrix multiplication via arithmetic progressions,''
{\it Proceedings of the 19th ACM Symposium on Theory of Computing},
1987, pp.~1--6.
\endrefs

% **** THIS ENDS THE EXAMPLES. DON'T DELETE THE FOLLOWING LINE:

\end{document}








