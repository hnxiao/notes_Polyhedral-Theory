
\documentclass{article}

\setlength{\oddsidemargin}{0.05 in}
\setlength{\evensidemargin}{-0.05 in}
\setlength{\topmargin}{-0.6 in}
\setlength{\textwidth}{6.5 in}
\setlength{\textheight}{9.5 in}
\setlength{\headsep}{0.25 in}
\setlength{\parskip}{0.1 in}

%
% ADD PACKAGES here:
\usepackage [usenames] {color}
\definecolor {infocolor} {rgb} {0.6,0.6,0.6}
\definecolor {steel blue}{rgb}{0.274510,0.509804,0.705882}
\everymath{\color{steel blue}}
%
\usepackage{IEEEtrantools}


\usepackage{amsmath,amsfonts,amssymb,enumerate,graphicx}

%
% The following commands set up the lecnum (lecture number)
% counter and make various numbering schemes work relative
% to the lecture number.
%
\newcounter{lecnum}
\renewcommand{\thepage}{\thelecnum-\arabic{page}}
\renewcommand{\thesection}{\thelecnum.\arabic{section}}
\renewcommand{\theequation}{\thelecnum.\arabic{equation}}
\renewcommand{\thefigure}{\thelecnum.\arabic{figure}}
\renewcommand{\thetable}{\thelecnum.\arabic{table}}

\DeclareMathOperator{\rk}{rk}

\newcommand{\tran}{^{\mbox{\tiny $\top$}}}
\newcommand{\tleq}{^{\mbox{\tiny $\leqslant$}}}
\newcommand{\teq}{^{\mbox{\tiny $=$}}}

%
% The following macro is used to generate the header.
%
\newcommand{\lecture}[2]{
   \pagestyle{myheadings}
   \thispagestyle{plain}
   \newpage
   \setcounter{lecnum}{#1}
   \setcounter{page}{1}
   \noindent
   \begin{center}
       \vbox{\vspace{2mm}
         \hbox {\leftline{\Large LECTURE #1: \hfill}}
         \vspace{3mm}
         \hbox {\leftline{\Large #2 \hfill}}
         \vspace{4mm}
         \hrule
         \vspace{3mm}
         \hbox to 6.5in { {{\large Polyhedra Theory}  \hfill July 2013} }
         \vspace{3mm}
        }
   \end{center}
   \markboth{LECTURE #1: #2}{LECTURE #1: #2}
   \pagenumbering{arabic}
   \vspace*{4mm}
}
%
% Convention for citations is authors' initials followed by the year.
% For example, to cite a paper by Leighton and Maggs you would type
% \cite{LM89}, and to cite a paper by Strassen you would type \cite{S69}.
% (To avoid bibliography problems, for now we redefine the \cite command.)
% Also commands that create a suitable format for the reference list.
\renewcommand{\cite}[1]{[#1]}
\def\beginrefs{\begin{list}%
        {[\arabic{equation}]}{\usecounter{equation}
         \setlength{\leftmargin}{2.0truecm}\setlength{\labelsep}{0.4truecm}%
         \setlength{\labelwidth}{1.6truecm}}}
\def\endrefs{\end{list}}
\def\bibentry#1{\item[\hbox{[#1]}]}

%Use this command for a figure; it puts a figure in wherever you want it.
%usage: \fig{NUMBER}{SPACE-IN-INCHES}{CAPTION}
\newcommand{\fig}[3]{
        	\vspace{#2}
			\begin{center}
			Figure \thelecnum.#1:~#3
			\end{center}
	}
% Use these for theorems, lemmas, proofs, etc.
\newtheorem{theorem}{Theorem}[lecnum]
\newtheorem{lemma}[theorem]{Lemma}
\newtheorem{proposition}[theorem]{Proposition}
\newtheorem{claim}[theorem]{Claim}
\newtheorem{corollary}[theorem]{Corollary}
\newtheorem{definition}[theorem]{Definition}
\newenvironment{proof}{{\it Proof.}}{ \hfill $\square$}

% **** IF YOU WANT TO DEFINE ADDITIONAL MACROS FOR YOURSELF, PUT THEM HERE:

\def\R{{\mathbb R}}
\def\Q{{\mathbb Q}}
\def\K{{\mathbb K}}

\begin{document}
%FILL IN THE RIGHT INFO.
%\lecture{**LECTURE-NUMBER**}{**DATE**}{**LECTURER**}{**SCRIBE**}
\lecture{8}{Doubly stochastic matrices}
%\footnotetext{These notes are partially based on those of Nigel Mansell.}

% **** YOUR NOTES GO HERE:

% Some general latex examples and examples making use of the
% macros follow.
%**** IN GENERAL, BE BRIEF. LONG SCRIBE NOTES, NO MATTER HOW WELL WRITTEN,
%**** ARE NEVER READ BY ANYBODY.

\section{Implicit equalities and redundant constraints}
A square matrix $A=(\alpha_{ij})_{n\times n}$ is called doubly stochastic if
\begin{IEEEeqnarray}{rCl}
\sum_{i=1}^n \alpha_{ij}=1 \IEEEyessubnumber \\
\sum_{j=1}^n \alpha_{ij}=1 \IEEEyessubnumber \\
\alpha_{ij}\geq 0 \IEEEyessubnumber
\end{IEEEeqnarray}

A permutation matrix is a $\{0,1\}$-matrix with exactly one 1 in each row and in each column.

The key idea here is to think of the $n\times n$ matrix $A$ given by the system $???$ above as a $n^2$-dimensional vector in $\R^{n^2}$. Then the system abvoe define a polytope of doubly stochastic matrices.

\begin{theorem}
Matrix $A$ is doubly stochastic if and only if $A$ is a convex combination of permutation matrices.
\end{theorem}
\begin{proof}
Sufficiency is obvious, as each permutation matrix is doubly stochastic.

Necessity is prove by induction on the order $n$ of $A$, the case $n=1$ being trivial. Consider the polytope $P$ defined by system $(???)$ above. Clearly, $P$ is a $n^2$-dimensional polytope of doubly stochastic matrices of order $n$. To prove the theorem, it suffices to show that each vertex of $P$ is convex combination of permutation matrices, since each point in this polytope is a convex combination of vertices of $P$. Let $A$ be a vertex of $P$. Then $n^2$ linearly independent constraints among $(???)$ are satisfied by $A$ with equality. Notice that the first $2n$ constraints in $???$ are linearly dependent, as the sum of all rows equals the sum of all columns. It follows that at least $n^2-2n+1$ of $\alpha_{ij}$ are $0$. So $A$ has a row with $n-1(>n^2-2n+1/n>n-2)$ $0$'s and one $1$. Without loss of generality, assume $\alpha_{11}=1$. So all the other entries in the first row and first column are $0$. Then deleting the first row and first column gives a doubly stochastic matrix of order $n-1$, which by inducton hyperthesis  is a convex combination of permutation matrices. Therefore $A$ itself is a convex combination of permutation matrices.
\end{proof}

The total unimodularity of matrix enables us to give an alternative proof.

\begin{proof}
Sufficiency is trivial.

Necessity is also to consider the polytope of doubly stochastic matrix defined by the system above. It suffices to show that each vertex of the polytope defined by system (???) is integral and thus a permutation matrix. However, the matrix $M$ defining system above is the node-edge incidence matrix of graph $K_{n,n}$.
$$M=\bordermatrix{~& \alpha_{11} & \alpha_{12} & \ldots & \alpha_{1n} & \ldots & \alpha_{n1} & \alpha_{n2} & \ldots & \alpha_{nn}\cr
                x_1 & 1 & 1 & \ldots & 1 & \ldots & 0 & 0 & \ldots & 0\cr
                x_2 & 0 & 0 & \ldots & 0 & \ldots & 0 & 0 & \ldots & 0\cr
                ~\vdots &   &   &   &  &   &   &   &   &  \cr
                x_n & 0 & 0 & \ldots & 0 & \ldots & 1 & 1 & \ldots & 1\cr
                y_1 & 1 & 0 & \ldots & 0 & \ldots & 1 & 0 & \ldots & 0\cr
                y_2 & 0 & 1 & \ldots & 0 & \ldots & 0 & 1 & \ldots & 0\cr
                ~\vdots &   &   &   &  &   &   &   &   &  \cr
                y_n & 0 & 0 & \ldots & 1 & \ldots & 0 & 0 & \ldots & 1\cr}
$$
Since such a matrix is totally unimodular, the theorem follows.
\end{proof}

\begin{corollary}
Each vertex of polytope of doubly stochastic matrices is a permutation matrix.
\end{corollary}

\section*{References}
\beginrefs
\bibentry{CW87}{\sc D.~Coppersmith} and {\sc S.~Winograd},
``Matrix multiplication via arithmetic progressions,''
{\it Proceedings of the 19th ACM Symposium on Theory of Computing},
1987, pp.~1--6.
\endrefs

% **** THIS ENDS THE EXAMPLES. DON'T DELETE THE FOLLOWING LINE:

\end{document}







