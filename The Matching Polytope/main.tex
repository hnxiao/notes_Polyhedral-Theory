
\documentclass{article}

\setlength{\oddsidemargin}{0.05 in}
\setlength{\evensidemargin}{-0.05 in}
\setlength{\topmargin}{-0.6 in}
\setlength{\textwidth}{6.5 in}
\setlength{\textheight}{9.5 in}
\setlength{\headsep}{0.25 in}
\setlength{\parskip}{0.1 in}

%
% ADD PACKAGES here:
\usepackage [usenames] {color}
\definecolor {infocolor} {rgb} {0.6,0.6,0.6}
\definecolor {steel blue}{rgb}{0.274510,0.509804,0.705882}
\everymath{\color{steel blue}}
%
\usepackage{IEEEtrantools}


\usepackage{amsmath,amsfonts,amssymb,enumerate,graphicx}

%
% The following commands set up the lecnum (lecture number)
% counter and make various numbering schemes work relative
% to the lecture number.
%
\newcounter{lecnum}
\renewcommand{\thepage}{\thelecnum-\arabic{page}}
\renewcommand{\thesection}{\thelecnum.\arabic{section}}
\renewcommand{\theequation}{\thelecnum.\arabic{equation}}
\renewcommand{\thefigure}{\thelecnum.\arabic{figure}}
\renewcommand{\thetable}{\thelecnum.\arabic{table}}

\DeclareMathOperator{\rk}{rk}

\newcommand{\tran}{^{\mbox{\tiny $\top$}}}
\newcommand{\tleq}{^{\mbox{\tiny $\leqslant$}}}
\newcommand{\teq}{^{\mbox{\tiny $=$}}}

%
% The following macro is used to generate the header.
%
\newcommand{\lecture}[2]{
   \pagestyle{myheadings}
   \thispagestyle{plain}
   \newpage
   \setcounter{lecnum}{#1}
   \setcounter{page}{1}
   \noindent
   \begin{center}
       \vbox{\vspace{2mm}
         \hbox {\leftline{\Large LECTURE #1: \hfill}}
         \vspace{3mm}
         \hbox {\leftline{\Large #2 \hfill}}
         \vspace{4mm}
         \hrule
         \vspace{3mm}
         \hbox to 6.5in { {{\large Polyhedra methods in combinatorial optimization}  \hfill July 2013} }
         \vspace{3mm}
        }
   \end{center}
   \markboth{LECTURE #1: #2}{LECTURE #1: #2}
   \pagenumbering{arabic}
   \vspace*{4mm}
}
%
% Convention for citations is authors' initials followed by the year.
% For example, to cite a paper by Leighton and Maggs you would type
% \cite{LM89}, and to cite a paper by Strassen you would type \cite{S69}.
% (To avoid bibliography problems, for now we redefine the \cite command.)
% Also commands that create a suitable format for the reference list.
\renewcommand{\cite}[1]{[#1]}
\def\beginrefs{\begin{list}%
        {[\arabic{equation}]}{\usecounter{equation}
         \setlength{\leftmargin}{2.0truecm}\setlength{\labelsep}{0.4truecm}%
         \setlength{\labelwidth}{1.6truecm}}}
\def\endrefs{\end{list}}
\def\bibentry#1{\item[\hbox{[#1]}]}

%Use this command for a figure; it puts a figure in wherever you want it.
%usage: \fig{NUMBER}{SPACE-IN-INCHES}{CAPTION}
\newcommand{\fig}[3]{
        	\vspace{#2}
			\begin{center}
			Figure \thelecnum.#1:~#3
			\end{center}
	}
% Use these for theorems, lemmas, proofs, etc.
\newtheorem{theorem}{Theorem}[lecnum]
\newtheorem{lemma}[theorem]{Lemma}
\newtheorem{proposition}[theorem]{Proposition}
\newtheorem{claim}[theorem]{Claim}
\newtheorem{corollary}[theorem]{Corollary}
\newtheorem{definition}[theorem]{Definition}
\newenvironment{proof}{{\it Proof.}}{ \hfill $\square$}

% **** IF YOU WANT TO DEFINE ADDITIONAL MACROS FOR YOURSELF, PUT THEM HERE:

\def\R{{\mathbb R}}
\def\Q{{\mathbb Q}}
\def\K{{\mathbb K}}

\begin{document}
%FILL IN THE RIGHT INFO.
%\lecture{**LECTURE-NUMBER**}{**DATE**}{**LECTURER**}{**SCRIBE**}
\lecture{8}{The Matching Polytope}
%\footnotetext{These notes are partially based on those of Nigel Mansell.}

% **** YOUR NOTES GO HERE:

% Some general latex examples and examples making use of the
% macros follow.
%**** IN GENERAL, BE BRIEF. LONG SCRIBE NOTES, NO MATTER HOW WELL WRITTEN,
%**** ARE NEVER READ BY ANYBODY.

\section{characteristic vectors}
For any $T \subseteq S$, the characteristic vector of $T$ is the $\{0,1\}^S$-vector satisfying
\[
x_e^T=
\begin{cases}
1       &  e\in T \\
0       &  \mbox{otherwise}
\end{cases}
\]

Let $G=(V,E)$ be a graph. The perfect matching polytope $P_{pm}(G)$ of $G$ is the convex hull of the characteristic vectors of perfect matching of $G$, and the matching polytope $P_m(G)$ of $G$ is the convex hull of the characteristic vectors of matchings of $G$.

Clearly, any characteristic vectors of a (perfect) matching is a vertex of the (perfect) matching polytope. The origin and standard unit vectors of $\R^E$ are characteristic vectors of matching. So $\dim(P_m(G))=\lvert E\rvert$. The dimension of $P_{pm}(G)$ is obviously smaller than $\lvert E\rvert$, as all vertices are contained in the hyperplane $\{x| \mathbf{1}^T x=\lvert V\rvert/2 \}$.

Now we already have the V-representation for (perfect) matching polytope, but we are more interested in the H-representation. Since inequality constraints enable us to solve some questions initiated by (based on) matching polytope by using linear programming. The following description is due to Jack Edmonds (1965). It sometimes says that this result initiated the use of polyhedral method in combinatorial optimization.

As usual, we start with bipartite graph which is much easier to handle. Notice that any vector $x$ of the perfect matching polytope $P_{pm}(G)$ satisfies:
\begin{subequations}
\begin{align}
       x_e & \geq 0, \mbox{~for~}  e\in E \\
       x(\delta(v)) &=1, \mbox{~for~}  v\in V
\end{align}
\end{subequations}
where...
The first condition (8.1a) is trivial. (8.1b) comes from the fact that each vertex $x$ is a convex combination of all vectices of $P_{pm}{G}$, i.e., $x(\delta(v))=\sum\lambda y(\delta(v))=\sum\lambda=1$, where $y$ stands for vertices of $P_{pm}(G)$.


\section*{References}
\beginrefs
\bibentry{CW87}{\sc D.~Coppersmith} and {\sc S.~Winograd},
``Matrix multiplication via arithmetic progressions,''
{\it Proceedings of the 19th ACM Symposium on Theory of Computing},
1987, pp.~1--6.
\endrefs

% **** THIS ENDS THE EXAMPLES. DON'T DELETE THE FOLLOWING LINE:

\end{document}







